\documentclass[10pt]{article}
\usepackage{moreverb}
\usepackage{algorithmic}
\usepackage{setspace} 
\usepackage{url}
\doublespacing

% Set Margins
\setlength{\textwidth}{6.5in}
\setlength{\oddsidemargin}{0in}
\hoffset=-.2in
\setlength{\textheight}{9.5in}
\setlength{\topmargin}{-1in}

\title{Poem Generation from Large Corpora}
\author{Nathaniel K Smith}

\begin{document}

\maketitle

\begin{abstract}
This paper presents a new approach to computer poetry, presenting an algorithm
that performs a directed cut-up generation technique on a large corpus. The
technical implementation of the algorithm and its associated modules will be
discussed in detail. To provide background and motivation, a survey of existing
computer poetry generation techniques is presented with an analysis of their
ability to produce interesting results. To support my analysis, I offer a
discussion of how the technical implementation of computer poetry generation
affects the production of interesting results. I argue that the new, presented
algorithm fills both a technical and creative gap within the existing computer
poetry landscape. Finally, opportunities for further research are discussed.
\end{abstract}

\section{Introduction}
\section{Background and Motivation}
\subsection{Two Categories}
\subsection{Existing Research}
\subsection{Analysis}
\subsubsection{Effect of Technical Implementation on Composed Work}
\subsubsection{The Interestingness Problem}
\section{Directed Cut-up Technique}
\subsection{Explication of Algorithm}
\subsection{Corpus Acquisition}
\subsection{Corpus Processing}
\subsection{Rules System}
\subsubsection{Input to Rules Translation}
\subsubsection{Rules to Query Translation}
\subsection{Final Composition}
\section{Results}
\subsection{Selected outputs}
\subsection{Defense of Technique}
\section{Further Research}

\bibliography{paper}
\bibliographystyle{apa}

\end{document}
