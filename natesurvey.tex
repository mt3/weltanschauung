\documentclass[10pt]{article}
\usepackage{moreverb}

% Set Margins
\setlength{\textwidth}{6.5in}
\setlength{\oddsidemargin}{0in}
\hoffset=-.2in
\setlength{\textheight}{9.5in}
\setlength{\topmargin}{-1in}

\title{Computer Poetry: A Survey}
\author{Nathaniel K Smith}

\begin{document}

\maketitle

\begin{abstract}
This paper explores the history, trends, and current state of Computer Poetry.
Examples of two major forms of computer mediated poetry creation will be
discussed: poetry where the computer is a generator of text, and poetry where
the computer is a filter or reassembler of an inputted text. Finally, this
paper will evaluate the extent to which computer poetry can evolve through the
utilization of Internet-based corpora.
\end{abstract}

\section{Introduction} 
The evolution and proliferation of computer poetry has been sluggish at best.
Charles O. Hartman suggests a simple explanations for this glacial trend: the
intersection of programmers and poets has historically been rather small
\cite{Hart96}. Thus, while architectures and programming languages have changed
drastically the fundamental techniques, algorithms, and motivations have not.

\subsection{History} 
Computer poetry traces its roots back to 1962 with the authorship of a program
called Auto-Beatnik. The software took in grammatical structures and vocabulary
and churned out stochastically rendered poems. It was not until 1984 that
another serious attempt at poetry generation was publicised with the release of
a work by the program RACTER called \emph{The Policeman's Beard is Half-
Constructed}. At 66 pages, \emph{Policeman's Beard} was a much larger effort
than the handful of short poems output by Auto-Beatnik. However, the generation
process was essentially the same idea: the program called upon built-in grammatical
templates and a large vocabulary of english words \cite{Chamb84}. 

In the same year appeared the Travesty Generator, written by literary scholar
Hugh Kenner and computer scientist Joseph O'Rourke. Distinct from its
contemporaries, the Travesty Generator produced 'poetry' without using built-in
lexicons. Instead, the Generator used an algorithm that, given some $n$,
reassembled an inputted text $t$ such that the result, $s$, contained all the
same $n$ length substrings in $t$. At $n = 1$ the program simply shuffled the
letters of $t$; but at about $n = 9$, $t$ begins to resemble a grammatically
correct rearrangment of the phrases in $s$ \cite{Hart96}. Thus, for varying
levels of $n$, new poetry could be produced from any input and this basic
algorithm without the need for any kind of external lexicon.

\subsection{Two Categories}
Auto-Beatnik and RACTER represent major seminal instances of one of the two
major categories of computer poetry that we are left with today. This category,
which I will refer to as \emph{Templatized} computer poetry, contains software
that relies on grammatical templates and a large vocabulary to create new
works. The software is essentially fed (or, in the case of the Prose system
discussed below, generating) sentences with blanks to fill in using a
dictionary.

The other category, which I will refer to as \emph{Generative}, contains
software which, like the Travesty Engine, relies only on input and an algorithm
to produce its work. At its simplest this category would include a basic
software designed only to, say, shuffle the characters or words in its input
but scales up to grammatical natural language generation in conversation with a
human user.

\section{Method 1}
\section{Method 2}
\section{Conclusion}

\bibliography{surv}
\bibliographystyle{apa}

\end{document}
