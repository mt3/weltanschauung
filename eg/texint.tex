\documentclass[10pt]{article}
\usepackage{moreverb}

% Set Margins
\setlength{\textwidth}{6.5in}
\setlength{\oddsidemargin}{0in}
\hoffset=-.2in
\setlength{\textheight}{9.5in}
\setlength{\topmargin}{-1in}

\title{Introduction to \LaTeX\ for Sophomore Seminar}
\author{Jennifer Ziebarth \\
(with slight modifications by Charlie Peck)}

\begin{document}

\maketitle

\section{What is \LaTeX?}
\LaTeX\ is a typesetting program designed especially for writing
mathematical text.  You've already seen several examples of
mathematical writing which was typset using either \LaTeX\ or its
cousin, \TeX: Most handouts you get from me, Mic, or Tim are written
in \LaTeX, and these days many mathematical books -- including {\em
Chapter Zero} -- are typset using some form of \TeX. Writing
mathematics in \LaTeX\ takes some getting used to, but the results are
sufficiently pretty that I think you'll agree it's time well-spent.

\LaTeX\ is equally suited to writing documents for computer science; articles, reports, papers, dissertations, and the like.  The learning 

\section{Creating and Compiling a \LaTeX\ document}
The first thing to keep in mind is that unlike a typical word
processor, what you see as you create your \LaTeX\ file is not what the
final product will look like.  Instead, you will use a {\em text
editor} to type text intermixed 
with various codes telling \LaTeX\ what to do.  Then you'll run your
text through a {\em compiler}, and view the result with a {\em viewer}.
Depending on how you do it, you might need to run a second program to
get a printable \verb|.pdf| file.
 
Schematically, this process looks like:

\begin{center}

\begin{tabular}{clc}
    Text Editor (\verb|emacs|) & & Viewer (\verb|xdvi|) \\
    \begin{boxedverbatim}
	$\frac{d}{dx} \int_a^x
	f(t) \; dt = f(x)$
    \end{boxedverbatim}
    & $\longrightarrow$ \verb|latex| $\longrightarrow$ &
    \fbox{\parbox[b]{1.25in}
    {$\frac{d}{dx}\int_a^x f(t) \; dt = f(x)$}} \\
    \hfill $\searrow$ & & $\downarrow$ \\
    & \verb|pdflatex| \hfill & \verb|dvipdfm| \\
    & \hfill $\searrow$ & $\downarrow$ \\
    && Printable \verb|.pdf| file (\verb|acroread|) \\
    &&     \fbox{\parbox[b]{1.25in}
    {$\frac{d}{dx}\int_{a}^{x} f(t) \; dt = f(x)$}} \\
\end{tabular}
\end{center}

On quark and the ACL machines this can be acheived with: 
\begin{verbatim}
codd$ latex paper.tex           # DVI output, viewed with xdvi 
codd$ pdflatex paper.tex        # PDF output, viewed with acroread
\end{verbatim}

\section{A simple \LaTeX\ document}
By way of example, on a following page is a simple \LaTeX\ file input
file and its output.  Here are
some things to notice about it.

\subsection{Preamble}
The first few lines of the document, prior to the
\verb|\begin{document}|, give \LaTeX\ some preliminary information
about the document, tell \LaTeX\ where to look for some of the commands
that will be used in the document.  The preamble also defines a few
commands to be used later.  Don't worry much about the preamble for
now.

\subsection{Beginnings and Endings}
The main part of the document begins with \verb|\begin{document}| and
ends with \verb|\end{document}|.  In between, there are several
smaller sections of the document that are set off with
\verb|\begin{something}| and \verb|\end{something}|.

\subsection{Mathematics}
The heart of \LaTeX\ is its ability to typeset mathematics.  Anything
mathematical is set off with either a single dollar sign \verb|$| at
the beginning and the end, or with a double dollar sign \verb|$$| at
the beginning and the end.  Mathematical symbols are represented by
commands such as \verb|\cup| or \verb|\in|.


\subsection{Graphics}
If you want to get fancy, you can include files created elsewhere into
a \LaTeX\ document.  The easiest way to do this is to save your picture
as an Encapsulated Postscript (\verb|.eps|) file.  Maple, in
particular, can export any graph it creates to an \verb|.eps| file. 
The lines in the preamble following \verb|%Figures| define a command,
\verb|\myfig|, for including figures.  The \verb|\myfig| command is
used at the end of the document.  

{\em Caution:} \verb|.eps| figures typically can't be seen when you
view the file with \verb|xdvi|, only when you look at the \verb|.pdf| 
file.

\section{A More Complicated \LaTeX\ document}
Following the simple \LaTeX\ file and its output, you'll find a more
complicated \LaTeX file and the output it produces.  Some things to
notice about the more complicated document:

\subsection{``Hidden'' preamble}
The preamble to the second document is actually pretty short, but
that's because the most complicated parts have been placed into
another file, \verb|macros.tex|.

\subsection{Sectioning}
This document is broken down into numbered sections.  The sections are
given names inside the \LaTeX\ file with the \verb|\label| command, so
they can be referred to by number with the \verb|\ref| command.
\LaTeX\ will keep track of the section number for you, even if you
insert a new section, causing everything to be re-numbered.
Similarly, theorems are numbered and can be referred to by number.

\subsection{Tables}
\LaTeX\ can make tables.  Within a table, separate entries with
\verb|&| and end a line with \verb|\\|.

\begin{center}
\begin{boxedverbatim}
\documentclass[10pt]{article}
\usepackage{amsmath,amsthm}
\usepackage{epsfig}

\newcommand{\myfig}[2]{\begin{figure}[htbp]\begin{center}
      {\scalebox{.4}{\includegraphics{#1.eps}}}
      \caption{#2}\label{fig:#1}
    \end{center}\end{figure}}

\newtheorem*{theorem}{Theorem}

\begin{document}

\begin{theorem}[2.4.2]
    Let $A$, $B$, and $C$ be sets.  Then
    $$A \cup (B \cap C) = (A \cup B) \cap (A \cup C).$$
\end{theorem}

\begin{proof}
    To show that $A \cup (B \cap C) = (A \cup B) \cap (A \cup C)$ we
    must show two things:
    
    \begin{align*}
	(\subseteq) & \qquad A \cup (B \cap C) \subseteq (A \cup B)
	\cap (A \cup C), \\
	(\supseteq) & \qquad (A \cup B) \cap (A \cup C) \subseteq A
	\cup (B \cap C).
    \end{align*}

    \begin{itemize}
	\item[$(\subseteq)$] 
	To prove the first inclusion, let $x \in A \cup (B \cap C)$.
	This means that $x \in A$ or $x \in (B \cap C)$.  If $x \in
	A$, then $x$ is in the union of $A$ with any other set.  So $x
	\in A \cup B$, and $x \in A \cup C$.  Likewise, if $x \in (B
	\cap C)$, then $x \in B$ and $x \in C$.  We can deduce from
	this that $x \in A \cup B$ and that $x \in A \cup C$.  In
	either case, $x$ is seen to be in $(A \cup B) \cap (A \cup
	C)$.  This concludes the proof that $A \cup (B \cap C)
	\subseteq (A \cup B) \cap (A \cup C)$.


	\item[$(\supseteq)$]
	Conversely, to prove the second inclusion, suppose that $x \in
	(A \cup B) \cap (A \cup C)$.  That is, $x \in (A \cup B)$ and
	$x \in (A \cup C)$.  We know in general, of course, that
	either $x \in A$ or $x \not\in A$.  If $x \in A$, then
	certainly $x \in A \cup (B \cap C).$ So suppose that $x
	\not\in A$.  In this case, $x$ must be in $B$ because $x \in A
	\cup B$.  Also we see that $x \in C$ since $x \in A \cup C$.
	We then conclude that $x \in B \cap C$.  Hence here, too, $x
	\in A \cup (B \cap C)$.  We have now shown that $(A \cup B)
	\cap (A \cup C) \subseteq A \cup (B \cap C)$.
    \end{itemize}
    
    Therefore, $A \cup (B \cap C) = (A \cup B) \cap (A \cup C)$.
\end{proof}

\myfig{venn}{Venn Diagram for 
$A \cup (B \cap C) = (A \cup B) \cap (A \cup C)$}

\end{document}
\end{boxedverbatim}
\end{center}




\end{document}
